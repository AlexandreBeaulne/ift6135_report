\documentclass{article}

\usepackage[final]{nips_2018}

\usepackage[utf8]{inputenc} % allow utf-8 input
\usepackage[T1]{fontenc}    % use 8-bit T1 fonts
\usepackage{hyperref}       % hyperlinks
\usepackage{url}            % simple URL typesetting
\usepackage{booktabs}       % professional-quality tables
\usepackage{amsfonts}       % blackboard math symbols
\usepackage{nicefrac}       % compact symbols for 1/2, etc.
\usepackage{microtype}      % microtypography
\usepackage{float}
\usepackage{natbib}
\usepackage{graphicx}
\usepackage{subfigure}
\usepackage{booktabs}

\usepackage{color}

\bibliographystyle{plainnat}
\setlength{\parskip}{0pt}

\def\defeq{\dot=}

\title{Analyzing the Role of Temporal Differencing in Deep Reinforcement Learning}

\author{
  Alexandre Beaulne \\
  \texttt{alex@mgnr.io} \\
  ID 20087309
  \And
  Amina Madzhun \\
  \texttt{amina.madzhun@umontreal.ca} \\
  ID 20052277
}

\begin{document}

\maketitle
\section{Introduction}
	The objective of our project was to reproduce and compare results 
    using different Deep Reinforcement Learning algorithms considered 
    by \citet{amiranashvili2018analyzing}. Since a huge amount of work is 
    done by the authors, we decided to focus on Atari
    video games.
    
    The choice of the paper for the course project is explained 
    by the wide range of applicability of Deep Learning
    including the use of deep neural networks architectures
    to solve more challenging problems as in deep RL 
    being at the junction between two machine learning approaches.
    
\subsection*{Background}
	The key idea of Reinforcement Learning (RL) is the learning
    from interaction with an environment through observations,
    actions and rewards. \\
    Consider standard RL problem over discrete time steps
    involving an episodic setup (starting an episode at $t=0$ and finishing it at
    terminal time step $t=T$).
    Let an agent be in a stochastic environment 
    with its current state $s_t$ at time step $t$.
    The agent can perform
    an action $a_t$ from a discrete set of actions.
    This action leads the environment to a new state $s_{t + 1}$, 
    and gives a reward $r_t$ to the agent or can finish the episode. \\
    The goal of the agent is to learn a policy $\pi(a_t|s_t)$
    to obtain the maximum expected return over time which is
    the total amount
    of reward that it can get from the current time step 
    to the end of the episode,
    with future rewards being reduced by discount factor $\gamma$ 
    for each time step. \\
    For a given policy $\pi$ the value function and the action-value function
    are defined as:
	\begin{eqnarray*}
    	V^\pi(s_t) \defeq \mathbb{E}_{\pi} [\hat{R}_t | s_t], &
        Q^\pi(s_t, a_t) \defeq \mathbb{E}_{\pi} [\hat{R}_t | s_t, a_t] 
	\end{eqnarray*}
    where $\hat{R}_t$ is cumulated discounted reward 
    $R^\gamma_t = \sum_{i=t}^T \gamma^{i-t} r_i$ \\
    or a truncated sum $R^\tau_t = \sum_{i=t}^{t+\tau} r_i$ 
    ($\tau$ in this case is fixed number of steps - horizon)
    \\
    Corresponding optimal functions are
    \begin{eqnarray*}
    	V^*(s_t) \defeq \max_{\pi}V^\pi(s_t), &
        Q^*(s_t, a_t) \defeq \max_{\pi} Q^\pi(s_t, a_t) 
	\end{eqnarray*}
    \textcolor{red}{short TD MC theory, Q-learning, mention DQN ??} 
    
    
\section{Paper Summary}
% Give a summary of the work, in your own words
	The authors want to investigate the role of Temporal Differencing (TD) 
    in modern deep Reinforcement Learning and observe the use of
    different techniques in controlled comparable experiments to evaluate the 
    advantages of using one approach versus another.
    They reproduce classic results as well as use 
    complex (3D) environments to test for tasks requiring perception,
    navigation and control, while separating influencing factors. 
    They discover that Monte Carlo estimation 
    (MC) can be feasible alternative and can be even competitive to TD, 
    in some cases better
    in perception and control of visually complex 3D environments 
    dealing with reward sparsity, delayed rewards etc.
    The work is discussed in more details further.

\subsection{Motivation}
% - What are the motivation and problem? Give some background.
	In RL setup numerous results in the past have proven
    that for low-dimensional feature representations and linear function
    approximators Temporal Difference learning outperforms 
    Monte Carlo estimation. However, in deep RL the spectrum of problems  
    has been broadened by using deep network architectures for function
    approximation. This allowed to handle high dimensionality inputs,
    extremely large state space and more sophisticated environments with partial 
    observability, still TD serves as the basis for
    most of developed algorithms. One of the problems in deep RL is that
    the choice of algorithm for given task is based on empirical results.
    
    % change citation
    \citet{DBLP:journals/corr/DosovitskiyK16} successfully developed a Monte Carlo
    prediction based approach (Direct Future Prediction) which outscored TD, 
    albeit the algorithm involved the use of specific concepts and
    components. The achieved results
    could induce the aspiration to study the existing solutions on standard
    problems within the framework of comparable experiments and motivate
    to explore the ways to combine advantages of existing approaches in deep RL.
    
\subsection{Proposed approach}
% - How do they propose to solve it?
\color{red}
	The more complicated model architectures become with developing
    new solutions, the more serious becomes the issue of defining causes 
    for perplexing performance
    or the correctness of results. 
    To surpass potential mistakes in conclusion for an analysis of algorithms,
    the authors try to separate the factors which could influence
    the performance:
    \begin{enumerate}
    \item complexity of the environment  \\
    Experiments on 1D to 3D environments
    \item algorithm configurations \\
    Consider three essential aspects: 
    \begin{itemize}
      \item TD or MC (infinite) varying rollout length in the
      update, -> n-step Q learning
      \item infinite vs finite horizon -> Q\_MC
      \item pure based vs A3C not to be limited in conclusions -> A3C from Mnih
    \end{itemize}
    \item environment properties \\
    consider different factoring by criterias: sparse, delayed rewards, terminal
    state, reward type, perceptual complexity
    \end{enumerate}
    
    + additional help/observations in terms of health gathering<...> problems

\color{black}
	

\section{Experiments}
% Discuss the experiments conducted in the paper. A good way to go about doing this is to think about what the advantages of the model are, and what might be a good way to demonstrate that before looking at what was attempted by the authors.

% - What are the authors trying to demonstrate with the experiments?
 
 \textcolor{red}{describe controlled experiments, VizDoom environments, perception}

% - Are there experiments that might demonstrate weaknesses/strengths that should be there but aren’t?
% - Are there any flaws in the experiments they run? What are they?
% \textcolor{red}{paper review comments}
% 	https://openreview.net/forum?id=HyiAuyb0b

\subsection{Motivation for the experiments}
\subsection{Reproducing the main results}
% Reproducing the results. Apart from the experimental details included in the paper, you should also include details that were not included.
% - What are some of the hyperparameters that were excluded from the paper? How important were they to get things working?
% - Are there technical details (missing or wrong equations, etc.) that were missing from the paper? If yes, what were they and what did you do to fill in the gap. Justify your choices.


\begin{table}
\centering
\label{table1}
\begin{tabular}{@{}l|c|cc}
\toprule
                                             & \#steps & Seaquest       & Pong \\ \midrule
    A3C (Mnih et al., 2016)                  & 80M     & 2300           & 11.4 \\ \midrule
	$Q_{MC}$ (Amiranashvili et al., 2017)    & 60M     & \textbf{12708} & -4.2 \\
	20-step $Q$ (Amiranashvili et al., 2017) & 60M     & 4276           & 8.9  \\
	20-step A3C (Amiranashvili et al., 2017) & 60M     & 2021           & 20.6 \\ \midrule
    20-step $Q$                              & 5M      &                &                 \\  
    20-step A3C                              & 5M      &                & \textbf{21}     \\ \bottomrule
\end{tabular}
\caption{Calibration against published results on standard environments.}
\end{table}

\section{Discussion}
% - How is the experiment aligned with proposed approach/analysis
% - Criticism of missing information, unmentioned drawbacks (if any)


\citet{mnih2015} \\
OpenAI's Gym \citep{gym} \\
\citep{pytorch} \\
\citet{amiranashvili2018analyzing} \\
\citet{pmlr-v48-mniha16}

\pagebreak

\small

\bibliography{main}

\end{document}



